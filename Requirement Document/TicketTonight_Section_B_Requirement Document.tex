

\documentclass{article}

\title{Requirements Document for Tickets Tonight App}  % Title

\author{Team members: \\ Matt Agra, Weijian Lin, Runhang Li, \\Zhetian Sun, Zhixuan Lai, Jeffery Ouyang \\ \\ Client: \\Ticketmaster} % Author name

\date{\today} % Date for the report
\usepackage{graphicx}
\begin{document}
	
	\maketitle % Insert the title, author and date
	

	\section{Introduction }
		\subsection{Purpose}
		 The purpose of this Software Requirement Specifications (SRS) document is to provide a description about the features and use cases of the Ticket Tonight App 1.0. It will cover all functional and nonfunctional requirements of the product, as well as its UI and technologies being used.
		 \subsection{Project Scope}
		 The goal of Ticket Tonight, as its name suggests, is to provide event recommendation, event search and ticket purchase service which are oriented to last minute decision. We use Ticketmaster’s own database to fetch corresponding events and infused with users affinity data, finally generate choices for events which are still available tonight.  Ticket Tonight will send notifications to inform users about events which are still available tonight, based on users’ location, preference, event date and budget, and lead users to ticket purchase page once they decide to go. 
		 
	\section{Overall Description}
		\subsection{Product Perspective}
		Ticket Tonight is a follow-on member of Ticketmaster’s mobile app for ticket sales. It relies on Ticketmaster‘s user database to provide users’ preference for event choice and previous activity records, along with events information such as event data, type, location, and ticket price to generate recommendation choices for users to choose. Then it leads the user to Ticketmaster’s own purchase system for ticket buying.
		\begin{center}
			\includegraphics[width=90mm]{/Users/runhang/Downloads/ticket.jpeg}
		\end{center}
		\subsection{User Classes and Characteristics}
		 There will be two main types of user classes for this app: event promoters and potential ticket buyers. Event promoters will be able to easily tell Ticketmaster of last-minute ticket availabilities through the app. The promoter can specify the number of tickets available and the price they’d like to set for those available tickets. Potential ticket buyers, which will be the majority of users, will be notified when an event would be a good match for them based on their Ticketmaster purchase history and preferences. Potential ticket buyers will be able to browse nearby events and filter the results by location and event type. 
		 \subsection{Operating Environment}
		 Initially, the Tickets Tonight app will be built for Apple’s iOS 8 mobile operating system, but will be supported on devices with Apple’s iOS 7.0 and later. It will support all devices running iOS 8, including the new iPhone 6, iPhone 6 Plus, iPad Air 2, and iPad mini 3.  The app will take advantage of these new display sizes and utilize the device’s location and TouchID sensors for a better user experience in receiving recommendations and completing ticket purchases. Given enough market demand, an Android version of the app will be developed. 
	
	\section{External Interface Requirements}
		\subsection{User Interfaces}
			Ticket tonight will follow iOS human interface guideline. Wireframes and mockups TBD.
		\subsection{Software Interfaces}
			Ticket tonight will fetch and query data from Ticketmaster server via a RESTful interface specified by Ticketmaster. On the server side, the backend system will store user data and transactional data in a dedicated server, which will have a RESTful interface as well. On the client side, the Ticket tonight app will achieve client side caching using NSCache, NSIncrementalStore and Core Data.
	
	\section{Other Nonfunctional Requirements}
		\subsection{Performance Requirements}
           The core feature of this application would require the user to be able to log in to their Ticketmaster account and utilizing their previous purchase history and favorites along with their geographical location and time, recommend the user relevant events that are happening in their proximity. The events recommended should be within a 10 mile radius to ensure the event is relevant for the user. The event starting time should be at least 3 hours prior to the event to ensure the user has ample time to plan for the event. For this purpose, the application will need to be able to access the Ticketmaster database for the user data, preferences and query data for upcoming events and this calculations for recommendations shall be relatively timely. In order to facilitate this, a separate database or server could be implemented to enable the update to be completed within 5 seconds. The algorithm for producing the recommendation will need to be efficient as to finish calculating and producing at least 10 recommendations within the 5 seconds update time. When the user favorites an artist or performer on the application, the information should be propagated to their Ticketmaster account as well for future calculations of recommendations for events and the propagation cannot exceed 10 seconds in order to preserve a seamless user experience when in situations where both the app is running and user is also browsing the website through a web browser. For the transaction phase, the application needs to be able to handle cases where the internet connection is unstable and when the transaction fails because of that. In such situations, the transaction should fail within 8 seconds and give the user feedback. If the internet connection is stable and the transaction is possible, the time for the user to receive feedback of completion should not exceed 5 seconds. The transactions need to be atomic in nature and provide the all-or-nothing attribute where the transaction is either completed or has completely no effect to prevent double transactions and charging or the user’s account.
		\subsection{Security Requirements}
           Information regarding the user’s transaction needs to be secure. The application has to secure the user’s credit card information in addition to the user’s identity, address and any additional information. The application’s transaction phase will be required to have protection against anything’s attempt at phishing the user’s information. When the transaction is complete, the transaction details should be secured and the transaction for the ticket from ticketmaster to the user needs to be secured as well. 
		 \subsection{Software Quality Attributes}
		    The major merit of this application is its recommendation of possible events that the user may be interested in. To achieve this, the recommendation has to be relevant and precise. In addition, the events need to be relatively reachable, meaning close in proximity and in time. Specifically, the recommended events need to fulfill the aforementioned time and distance requirement of being 3 to 24 hours prior to the event and within a 10 mile radius of the user’s current location or the user’s registered location. The user should not be alerted or seeing events that they are not interested in or that have already expired. The graphical user interface of this application should be designed to be user-friendly as its priority to ensure the user having a satisfying experience. The user application response time for any action cannot exceed half a second. The recommendations automatic push notifications should be not intrusive and relevant. If the user desires to, the push notification should be allowed to be turned off. Visual cues and notifications will be used as feedbacks for the users to inform them that the transaction has been completed or that their transaction may have failed due to unstable internet connection. In cases of stable internet connection, the transaction complete feedback arrival time cannot exceed 5 seconds. In cases of unstable internet connection, the transaction failure feedback arrival time cannot exceed 8 seconds. Furthermore, the user form of the ticket given to the user should be also viewable and usable on the mobile device.
		 
\end{document}